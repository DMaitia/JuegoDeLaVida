\documentclass[10pt,a4paper]{article}
\usepackage[latin1]{inputenc}
\usepackage[spanish]{babel}
\usepackage{amsmath}
\usepackage{amsfonts}
\usepackage{amssymb}
\usepackage{graphicx}
\usepackage{listings}
\usepackage{vmargin}
\setpapersize{A4}
\setmargins
{2.5cm}       % margen izquierdo
{1.5cm} % margen superior
{16.5cm}% anchura del texto
{23.42cm} % altura del texto
{10pt} % altura de los encabezados
{1cm} % espacio entre el texto y los encabezados
{0pt} % altura del pie de p�gina
{2cm}% espacio entre el texto y el pie de p�gina


\begin{document}
\begin{figure}
\centering
\includegraphics[width=1.0\linewidth, height=0.23\textheight]{FIUBA_ALTA}
\label{fig:FIUBA_ALTA}
\end{figure}

	\begin{titlepage}
		\centering
		\vspace{1cm}
		{\scshape\Large Organizaci�n de Computadoras (66.20)\par}
		\vspace{1.5cm}
		{\huge\bfseries Conjunto de Instrucciones MIPS \par}
		\vspace{2cm}
		{\Large\itshape Cristian Gonz�lez 94719\par}
		\vspace{2cm}
		{\Large\itshape Darius Maitia 95436 \par}
	\end{titlepage}
	
	
	
	
	
	
	%�ndice.
	\tableofcontents
	\newpage
\section{Introducci�n}
El \textquotedblleft Juego de la Vida\textquotedblright  de Conway es un aut�mata celular, dise�ado por el
matem�tico brit�nico John Conway en 1970. Se trata b�sicamente de una grilla en principio infinita, en cada una de cuyas celdas puede haber un organismo vivo, en cada iteraci�n del juego, se va redefiniendo si un organismo (una celda de la grilla) cambia de estado (viva o muerta). \\
Este cambio de estado depende de sus vecinos (las posiciones alrededor de este) que define con ciertas reglas. Para eso se nos pide implementar una funci�n \texttt{vecinos()} que observa los vecinos y nos devuelve la cantidad de vecinos \textquotedblleft vivos\textquotedblright.\\
Esta funcionalidad ser� implementada en Assembler MIPS 32, pero tambi�n nuestro programa contara con su versi�n implementada totalmente en C.    
\subsection{Instrucciones}
Ac� se describe los pasos para ejecutar el juego. Se cuenta con los archivos:
\begin{itemize}
	\item JuegoDeLaVida.c
	\item vecinos.S
\end{itemize}
Los primeros casos consiste en compilar ambos archivos:
\begin{itemize}
	\item \texttt{gcc -ggdb -Wall -c vecinos.S}
	\item \texttt{gcc -ggdb -Wall -c JuegoDeLaVida.c}
\end{itemize}
Se procede a linkear y crear un ejecutable.
\begin{itemize}
	\item \texttt{gcc vecinos.o JuegoDeLaVida.o -o conway}
\end{itemize}

Para la ejecuci�n del programa se debe realizar la siguiente linea:\\

\texttt{conway i M N inputfile [-o outputprefix]}
\newpage
\subsection{C�digo Assembler}
\lstinputlisting{vecinos.S}
\begin{lstlisting}

\end{lstlisting}

\section{Documentacion de Corridas}	
Contenido del archivo \texttt{pento}:\\
\texttt{cat pento}
\texttt{\\
3 5\\
3 6\\
4 4\\
4 5\\
5 5\\
}
\newpage

Contenido del archivo \texttt{sapo}:\\
\texttt{cat sapo}
\texttt{\\
5 3\\
5 4\\
5 5\\
4 4\\
4 5\\
4 6\\
}\\

Contenido del archivo \texttt{glider}:\\
\texttt{cat glider}
\texttt{\\
5 3\\
5 4\\
5 5\\
3 4\\
4 5\\
}
\subsection{Corrida glider}
Comando ejecutado:\\
\texttt{./conway 10 20 20 glider -o salida}\\
Salida stdin:
\texttt{\\
	Grabando salida\_1.pbm\\
	Listo\\
	Grabando salida\_2.pbm\\
	Listo\\
	Grabando salida\_3.pbm\\
	Listo\\
	Grabando salida\_4.pbm\\
	Listo\\
	Grabando salida\_5.pbm\\
	Listo\\
	Grabando salida\_6.pbm\\
	Listo\\
	Grabando salida\_7.pbm\\
	Listo\\
	Grabando salida\_8.pbm\\
	Listo\\
	Grabando salida\_9.pbm\\
	Listo\\
	Grabando salida\_10.pbm\\
	Listo}
\newpage
Archivos:\\
\texttt{cat salida\_1.pbm\\
	}
\texttt{P1 20 20\\
	0 0 0 0 0 0 0 0 0 0 0 0 0 0 0 0 0 0 0 0\\ 
	0 0 0 0 0 0 0 0 0 0 0 0 0 0 0 0 0 0 0 0 \\
	0 0 0 0 0 0 0 0 0 0 0 0 0 0 0 0 0 0 0 0\\ 
	0 0 0 0 1 0 0 0 0 0 0 0 0 0 0 0 0 0 0 0\\ 
	0 0 0 0 0 1 0 0 0 0 0 0 0 0 0 0 0 0 0 0 \\
	0 0 0 1 1 1 0 0 0 0 0 0 0 0 0 0 0 0 0 0 \\
	0 0 0 0 0 0 0 0 0 0 0 0 0 0 0 0 0 0 0 0 \\
	0 0 0 0 0 0 0 0 0 0 0 0 0 0 0 0 0 0 0 0 \\
	0 0 0 0 0 0 0 0 0 0 0 0 0 0 0 0 0 0 0 0 \\
	0 0 0 0 0 0 0 0 0 0 0 0 0 0 0 0 0 0 0 0 \\
	0 0 0 0 0 0 0 0 0 0 0 0 0 0 0 0 0 0 0 0 \\
	0 0 0 0 0 0 0 0 0 0 0 0 0 0 0 0 0 0 0 0\\
	0 0 0 0 0 0 0 0 0 0 0 0 0 0 0 0 0 0 0 0 \\
	0 0 0 0 0 0 0 0 0 0 0 0 0 0 0 0 0 0 0 0 \\
	0 0 0 0 0 0 0 0 0 0 0 0 0 0 0 0 0 0 0 0 \\
	0 0 0 0 0 0 0 0 0 0 0 0 0 0 0 0 0 0 0 0 \\
	0 0 0 0 0 0 0 0 0 0 0 0 0 0 0 0 0 0 0 0 \\
	0 0 0 0 0 0 0 0 0 0 0 0 0 0 0 0 0 0 0 0 \\
	0 0 0 0 0 0 0 0 0 0 0 0 0 0 0 0 0 0 0 0 \\
	0 0 0 0 0 0 0 0 0 0 0 0 0 0 0 0 0 0 0 0 \\
	}
	
\texttt{cat salida\_5.pbm\\
}
\texttt{P1 20 20\\
	0 0 0 0 0 0 0 0 0 0 0 0 0 0 0 0 0 0 0 0\\
	0 0 0 0 0 0 0 0 0 0 0 0 0 0 0 0 0 0 0 0 \\
	0 0 0 0 0 0 0 0 0 0 0 0 0 0 0 0 0 0 0 0 \\
	0 0 0 0 0 0 0 0 0 0 0 0 0 0 0 0 0 0 0 0 \\
	0 0 0 0 0 1 0 0 0 0 0 0 0 0 0 0 0 0 0 0 \\
	0 0 0 0 0 0 1 0 0 0 0 0 0 0 0 0 0 0 0 0 \\
	0 0 0 0 1 1 1 0 0 0 0 0 0 0 0 0 0 0 0 0 \\
	0 0 0 0 0 0 0 0 0 0 0 0 0 0 0 0 0 0 0 0 \\
	0 0 0 0 0 0 0 0 0 0 0 0 0 0 0 0 0 0 0 0 \\
	0 0 0 0 0 0 0 0 0 0 0 0 0 0 0 0 0 0 0 0 \\
	0 0 0 0 0 0 0 0 0 0 0 0 0 0 0 0 0 0 0 0 \\
	0 0 0 0 0 0 0 0 0 0 0 0 0 0 0 0 0 0 0 0 \\
	0 0 0 0 0 0 0 0 0 0 0 0 0 0 0 0 0 0 0 0 \\
	0 0 0 0 0 0 0 0 0 0 0 0 0 0 0 0 0 0 0 0 \\
	0 0 0 0 0 0 0 0 0 0 0 0 0 0 0 0 0 0 0 0 \\
	0 0 0 0 0 0 0 0 0 0 0 0 0 0 0 0 0 0 0 0 \\
	0 0 0 0 0 0 0 0 0 0 0 0 0 0 0 0 0 0 0 0 \\
	0 0 0 0 0 0 0 0 0 0 0 0 0 0 0 0 0 0 0 0 \\
	0 0 0 0 0 0 0 0 0 0 0 0 0 0 0 0 0 0 0 0 \\
	0 0 0 0 0 0 0 0 0 0 0 0 0 0 0 0 0 0 0 0 \\
	}
\texttt{cat salida\_10.pbm\\
}
\texttt{P1 20 20\\
0 0 0 0 0 0 0 0 0 0 0 0 0 0 0 0 0 0 0 0\\
0 0 0 0 0 0 0 0 0 0 0 0 0 0 0 0 0 0 0 0 \\
0 0 0 0 0 0 0 0 0 0 0 0 0 0 0 0 0 0 0 0 \\
0 0 0 0 0 0 0 0 0 0 0 0 0 0 0 0 0 0 0 0 \\
0 0 0 0 0 0 0 0 0 0 0 0 0 0 0 0 0 0 0 0 \\
0 0 0 0 0 0 0 0 0 0 0 0 0 0 0 0 0 0 0 0 \\
0 0 0 0 0 1 0 1 0 0 0 0 0 0 0 0 0 0 0 0 \\
0 0 0 0 0 0 1 1 0 0 0 0 0 0 0 0 0 0 0 0 \\
0 0 0 0 0 0 1 0 0 0 0 0 0 0 0 0 0 0 0 0 \\
0 0 0 0 0 0 0 0 0 0 0 0 0 0 0 0 0 0 0 0 \\
0 0 0 0 0 0 0 0 0 0 0 0 0 0 0 0 0 0 0 0 \\
0 0 0 0 0 0 0 0 0 0 0 0 0 0 0 0 0 0 0 0 \\
0 0 0 0 0 0 0 0 0 0 0 0 0 0 0 0 0 0 0 0 \\
0 0 0 0 0 0 0 0 0 0 0 0 0 0 0 0 0 0 0 0 \\
0 0 0 0 0 0 0 0 0 0 0 0 0 0 0 0 0 0 0 0 \\
0 0 0 0 0 0 0 0 0 0 0 0 0 0 0 0 0 0 0 0 \\
0 0 0 0 0 0 0 0 0 0 0 0 0 0 0 0 0 0 0 0 \\
0 0 0 0 0 0 0 0 0 0 0 0 0 0 0 0 0 0 0 0 \\
0 0 0 0 0 0 0 0 0 0 0 0 0 0 0 0 0 0 0 0 \\
0 0 0 0 0 0 0 0 0 0 0 0 0 0 0 0 0 0 0 0 \\
}
\subsection{Corrida sapo}
Comando ejecutado:\\
\texttt{./conway 10 20 20 sapo -o salida\_sapo}\\
Salida stdin:
\texttt{\\
	Grabando salida\_sapo\_1.pbm\\
	Listo\\
	Grabando salida\_sapo\_2.pbm\\
	Listo\\
	Grabando salida\_sapo\_3.pbm\\
	Listo\\
	Grabando salida\_sapo\_4.pbm\\
	Listo\\
	Grabando salida\_sapo\_5.pbm\\
	Listo\\
	Grabando salida\_sapo\_6.pbm\\
	Listo\\
	Grabando salida\_sapo\_7.pbm\\
	Listo\\
	Grabando salida\_sapo\_8.pbm\\
	Listo\\
	Grabando salida\_sapo\_9.pbm\\
	Listo\\
	Grabando salida\_sapo\_10.pbm\\
	Listo}
Archivos:\\
\texttt{cat salida\_1.pbm\\
}
\texttt{P1 20 20\\
P1 20 20
0 0 0 0 0 0 0 0 0 0 0 0 0 0 0 0 0 0 0 0 \\
0 0 0 0 0 0 0 0 0 0 0 0 0 0 0 0 0 0 0 0 \\
0 0 0 0 0 0 0 0 0 0 0 0 0 0 0 0 0 0 0 0 \\
0 0 0 0 1 0 0 0 0 0 0 0 0 0 0 0 0 0 0 0 \\
0 0 0 0 0 1 0 0 0 0 0 0 0 0 0 0 0 0 0 0 \\
0 0 0 1 1 1 0 0 0 0 0 0 0 0 0 0 0 0 0 0 \\
0 0 0 0 0 0 0 0 0 0 0 0 0 0 0 0 0 0 0 0 \\
0 0 0 0 0 0 0 0 0 0 0 0 0 0 0 0 0 0 0 0 \\
0 0 0 0 0 0 0 0 0 0 0 0 0 0 0 0 0 0 0 0 \\
0 0 0 0 0 0 0 0 0 0 0 0 0 0 0 0 0 0 0 0 \\
0 0 0 0 0 0 0 0 0 0 0 0 0 0 0 0 0 0 0 0 \\
0 0 0 0 0 0 0 0 0 0 0 0 0 0 0 0 0 0 0 0 \\
0 0 0 0 0 0 0 0 0 0 0 0 0 0 0 0 0 0 0 0 \\
0 0 0 0 0 0 0 0 0 0 0 0 0 0 0 0 0 0 0 0 \\
0 0 0 0 0 0 0 0 0 0 0 0 0 0 0 0 0 0 0 0 \\
0 0 0 0 0 0 0 0 0 0 0 0 0 0 0 0 0 0 0 0 \\
0 0 0 0 0 0 0 0 0 0 0 0 0 0 0 0 0 0 0 0 \\
0 0 0 0 0 0 0 0 0 0 0 0 0 0 0 0 0 0 0 0 \\
0 0 0 0 0 0 0 0 0 0 0 0 0 0 0 0 0 0 0 0 \\
0 0 0 0 0 0 0 0 0 0 0 0 0 0 0 0 0 0 0 0 \\
}

\texttt{cat salida\_sapo\_5.pbm\\
}
\texttt{P1 20 20\\
0 0 0 0 0 0 0 0 0 0 0 0 0 0 0 0 0 0 0 0\\ 
0 0 0 0 0 0 0 0 0 0 0 0 0 0 0 0 0 0 0 0 \\
0 0 0 0 0 0 0 0 0 0 0 0 0 0 0 0 0 0 0 0 \\
0 0 0 0 0 0 0 0 0 0 0 0 0 0 0 0 0 0 0 0 \\
0 0 0 0 0 1 0 0 0 0 0 0 0 0 0 0 0 0 0 0 \\
0 0 0 0 0 0 1 0 0 0 0 0 0 0 0 0 0 0 0 0 \\
0 0 0 0 1 1 1 0 0 0 0 0 0 0 0 0 0 0 0 0 \\
0 0 0 0 0 0 0 0 0 0 0 0 0 0 0 0 0 0 0 0 \\
0 0 0 0 0 0 0 0 0 0 0 0 0 0 0 0 0 0 0 0 \\
0 0 0 0 0 0 0 0 0 0 0 0 0 0 0 0 0 0 0 0 \\
0 0 0 0 0 0 0 0 0 0 0 0 0 0 0 0 0 0 0 0 \\
0 0 0 0 0 0 0 0 0 0 0 0 0 0 0 0 0 0 0 0 \\
0 0 0 0 0 0 0 0 0 0 0 0 0 0 0 0 0 0 0 0 \\
0 0 0 0 0 0 0 0 0 0 0 0 0 0 0 0 0 0 0 0 \\
0 0 0 0 0 0 0 0 0 0 0 0 0 0 0 0 0 0 0 0 \\
0 0 0 0 0 0 0 0 0 0 0 0 0 0 0 0 0 0 0 0 \\
0 0 0 0 0 0 0 0 0 0 0 0 0 0 0 0 0 0 0 0 \\
0 0 0 0 0 0 0 0 0 0 0 0 0 0 0 0 0 0 0 0 \\
0 0 0 0 0 0 0 0 0 0 0 0 0 0 0 0 0 0 0 0 \\
0 0 0 0 0 0 0 0 0 0 0 0 0 0 0 0 0 0 0 0\\
}
\texttt{cat salida\_sapo\_10.pbm\\
}
\texttt{P1 20 20\\
0 0 0 0 0 0 0 0 0 0 0 0 0 0 0 0 0 0 0 0 \\
0 0 0 0 0 0 0 0 0 0 0 0 0 0 0 0 0 0 0 0 \\
0 0 0 0 0 0 0 0 0 0 0 0 0 0 0 0 0 0 0 0 \\
0 0 0 0 0 0 0 0 0 0 0 0 0 0 0 0 0 0 0 0 \\
0 0 0 0 0 0 0 0 0 0 0 0 0 0 0 0 0 0 0 0 \\
0 0 0 0 0 0 0 0 0 0 0 0 0 0 0 0 0 0 0 0 \\
0 0 0 0 0 1 0 1 0 0 0 0 0 0 0 0 0 0 0 0 \\
0 0 0 0 0 0 1 1 0 0 0 0 0 0 0 0 0 0 0 0 \\
0 0 0 0 0 0 1 0 0 0 0 0 0 0 0 0 0 0 0 0 \\
0 0 0 0 0 0 0 0 0 0 0 0 0 0 0 0 0 0 0 0 \\
0 0 0 0 0 0 0 0 0 0 0 0 0 0 0 0 0 0 0 0 \\
0 0 0 0 0 0 0 0 0 0 0 0 0 0 0 0 0 0 0 0 \\
0 0 0 0 0 0 0 0 0 0 0 0 0 0 0 0 0 0 0 0 \\
0 0 0 0 0 0 0 0 0 0 0 0 0 0 0 0 0 0 0 0 \\
0 0 0 0 0 0 0 0 0 0 0 0 0 0 0 0 0 0 0 0 \\
0 0 0 0 0 0 0 0 0 0 0 0 0 0 0 0 0 0 0 0 \\
0 0 0 0 0 0 0 0 0 0 0 0 0 0 0 0 0 0 0 0 \\
0 0 0 0 0 0 0 0 0 0 0 0 0 0 0 0 0 0 0 0 \\
0 0 0 0 0 0 0 0 0 0 0 0 0 0 0 0 0 0 0 0 \\
0 0 0 0 0 0 0 0 0 0 0 0 0 0 0 0 0 0 0 0 \\
}

\subsection{Corrida pento}
Comando ejecutado:\\
\texttt{./conway 10 20 20 pento -o salida\_pento}\\
Salida stdin:
\texttt{\\
	Grabando salida\_pento\_1.pbm\\
	Listo\\
	Grabando salida\_pento\_2.pbm\\
	Listo\\
	Grabando salida\_pento\_3.pbm\\
	Listo\\
	Grabando salida\_pento\_4.pbm\\
	Listo\\
	Grabando salida\_pento\_5.pbm\\
	Listo\\
	Grabando salida\_pento\_6.pbm\\
	Listo\\
	Grabando salida\_pento\_7.pbm\\
	Listo\\
	Grabando salida\_pento\_8.pbm\\
	Listo\\
	Grabando salida\_pento\_9.pbm\\
	Listo\\
	Grabando salida\_pento\_10.pbm\\
	Listo}
Archivos:\\
\texttt{cat salida\_1.pbm\\
}
\texttt{P1 20 20\\
0 0 0 0 0 0 0 0 0 0 0 0 0 0 0 0 0 0 0 0 \\
0 0 0 0 0 0 0 0 0 0 0 0 0 0 0 0 0 0 0 0 \\
0 0 0 0 0 0 0 0 0 0 0 0 0 0 0 0 0 0 0 0 \\
0 0 0 0 0 1 1 0 0 0 0 0 0 0 0 0 0 0 0 0 \\
0 0 0 0 1 1 0 0 0 0 0 0 0 0 0 0 0 0 0 0 \\
0 0 0 0 0 1 0 0 0 0 0 0 0 0 0 0 0 0 0 0 \\
0 0 0 0 0 0 0 0 0 0 0 0 0 0 0 0 0 0 0 0 \\
0 0 0 0 0 0 0 0 0 0 0 0 0 0 0 0 0 0 0 0 \\
0 0 0 0 0 0 0 0 0 0 0 0 0 0 0 0 0 0 0 0 \\
0 0 0 0 0 0 0 0 0 0 0 0 0 0 0 0 0 0 0 0 \\
0 0 0 0 0 0 0 0 0 0 0 0 0 0 0 0 0 0 0 0 \\
0 0 0 0 0 0 0 0 0 0 0 0 0 0 0 0 0 0 0 0 \\
0 0 0 0 0 0 0 0 0 0 0 0 0 0 0 0 0 0 0 0 \\
0 0 0 0 0 0 0 0 0 0 0 0 0 0 0 0 0 0 0 0 \\
0 0 0 0 0 0 0 0 0 0 0 0 0 0 0 0 0 0 0 0 \\
0 0 0 0 0 0 0 0 0 0 0 0 0 0 0 0 0 0 0 0 \\
0 0 0 0 0 0 0 0 0 0 0 0 0 0 0 0 0 0 0 0 \\
0 0 0 0 0 0 0 0 0 0 0 0 0 0 0 0 0 0 0 0 \\
0 0 0 0 0 0 0 0 0 0 0 0 0 0 0 0 0 0 0 0 \\
0 0 0 0 0 0 0 0 0 0 0 0 0 0 0 0 0 0 0 0\\
}

\texttt{cat salida\_pento\_5.pbm\\
}
\texttt{P1 20 20\\
0 0 0 0 0 0 0 0 0 0 0 0 0 0 0 0 0 0 0 0 \\
0 0 0 0 0 0 0 0 0 0 0 0 0 0 0 0 0 0 0 0 \\
0 0 0 0 1 0 1 0 0 0 0 0 0 0 0 0 0 0 0 0 \\
0 0 0 1 0 0 1 0 0 0 0 0 0 0 0 0 0 0 0 0 \\
0 0 0 1 0 0 1 0 0 0 0 0 0 0 0 0 0 0 0 0 \\
0 0 0 0 1 1 0 0 0 0 0 0 0 0 0 0 0 0 0 0 \\
0 0 0 0 0 0 0 0 0 0 0 0 0 0 0 0 0 0 0 0 \\
0 0 0 0 0 0 0 0 0 0 0 0 0 0 0 0 0 0 0 0 \\
0 0 0 0 0 0 0 0 0 0 0 0 0 0 0 0 0 0 0 0 \\
0 0 0 0 0 0 0 0 0 0 0 0 0 0 0 0 0 0 0 0 \\
0 0 0 0 0 0 0 0 0 0 0 0 0 0 0 0 0 0 0 0 \\
0 0 0 0 0 0 0 0 0 0 0 0 0 0 0 0 0 0 0 0 \\
0 0 0 0 0 0 0 0 0 0 0 0 0 0 0 0 0 0 0 0 \\
0 0 0 0 0 0 0 0 0 0 0 0 0 0 0 0 0 0 0 0 \\
0 0 0 0 0 0 0 0 0 0 0 0 0 0 0 0 0 0 0 0 \\
0 0 0 0 0 0 0 0 0 0 0 0 0 0 0 0 0 0 0 0 \\
0 0 0 0 0 0 0 0 0 0 0 0 0 0 0 0 0 0 0 0 \\
0 0 0 0 0 0 0 0 0 0 0 0 0 0 0 0 0 0 0 0 \\
0 0 0 0 0 0 0 0 0 0 0 0 0 0 0 0 0 0 0 0 \\
0 0 0 0 0 0 0 0 0 0 0 0 0 0 0 0 0 0 0 0\\
}\\

\texttt{cat salida\_pento\_10.pbm\\
}
\texttt{P1 20 20\\
0 0 0 0 0 1 0 0 0 0 0 0 0 0 0 0 0 0 0 0 \\
0 0 0 1 0 0 1 0 0 0 0 0 0 0 0 0 0 0 0 0 \\
0 0 1 0 0 0 0 0 0 0 0 0 0 0 0 0 0 0 0 0 \\
0 0 1 0 0 0 0 1 0 0 0 0 0 0 0 0 0 0 0 0 \\
0 0 1 1 0 0 0 0 0 0 0 0 0 0 0 0 0 0 0 0 \\
0 0 0 0 0 0 0 0 0 0 0 0 0 0 0 0 0 0 0 0 \\
0 0 0 0 1 1 1 0 0 0 0 0 0 0 0 0 0 0 0 0 \\
0 0 0 0 0 0 0 0 0 0 0 0 0 0 0 0 0 0 0 0 \\
0 0 0 0 0 0 0 0 0 0 0 0 0 0 0 0 0 0 0 0 \\
0 0 0 0 0 0 0 0 0 0 0 0 0 0 0 0 0 0 0 0 \\
0 0 0 0 0 0 0 0 0 0 0 0 0 0 0 0 0 0 0 0 \\
0 0 0 0 0 0 0 0 0 0 0 0 0 0 0 0 0 0 0 0 \\
0 0 0 0 0 0 0 0 0 0 0 0 0 0 0 0 0 0 0 0 \\
0 0 0 0 0 0 0 0 0 0 0 0 0 0 0 0 0 0 0 0 \\
0 0 0 0 0 0 0 0 0 0 0 0 0 0 0 0 0 0 0 0 \\
0 0 0 0 0 0 0 0 0 0 0 0 0 0 0 0 0 0 0 0 \\
0 0 0 0 0 0 0 0 0 0 0 0 0 0 0 0 0 0 0 0 \\
0 0 0 0 0 0 0 0 0 0 0 0 0 0 0 0 0 0 0 0 \\
0 0 0 0 0 0 0 0 0 0 0 0 0 0 0 0 0 0 0 0 \\
0 0 0 0 0 0 0 0 0 0 0 0 0 0 0 0 0 0 0 0 \\
}


En caso de que no especifique archivo de salida:\\
\texttt{./conway 10 20 20 pento\\
Grabando pento\_1.pbm\\
Listo\\
Grabando pento\_2.pbm\\
Listo\\
Grabando pento\_3.pbm\\
Listo\\
Grabando pento\_4.pbm\\
Listo\\
Grabando pento\_5.pbm\\
Listo\\
Grabando pento\_6.pbm\\
Listo\\
Grabando pento\_7.pbm\\
Listo\\
Grabando pento\_8.pbm\\
Listo\\
Grabando pento\_9.pbm\\
Listo\\
Grabando pento\_10.pbm\\
Listo\\
}
\section{Definiciones del Juego}
\begin{itemize}
	\item Se realizo el juego de tal forma que si un vecino de la posici�n i,j est� fuera de rango de nuestra grilla, este vecino est� muerto.
\end{itemize}
\section{Stack}
\lstinputlisting{stack.txt}
\begin{lstlisting}

\end{lstlisting}
\section{Conclusiones}
Assembly representa un interesante recurso de programaci�n, en el que se hace necesario el conocimiento del funcionamiento interno del procesador as� como de el armado del stack entre funciones llamadas y llamadoras. \\
Concretamente en este trabajo pr�ctico, dicho punto fue un paso obligado debido a un quinto argumento de la funci�n vecinos que exed�a los registros de argumentos a0-a3. Para ello la funci�n llamadora actualizarMatriz reserv� en stack espacio adicional para guardar la variable supelementaria de manera tal que quede acomodada para la funcion llamada que a su vez reserva memoria adicional en el stack para guardar las variables que sean necesarias (en este caso �nicamente para guardar el $fp y $gp puesto que era una funci�n hoja). 
	
\end{document}
